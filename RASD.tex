\documentclass{article}
\usepackage{graphicx} % Required for inserting images

\begin{document}

% Title Page
\title{RASD Document}
\author{Alessandro Salvatore, Erdal Yalçın, Leonardo Ratti}
\date{Academic Year: 2024-25}
\maketitle

% Table of Contents
\newpage
\tableofcontents

% Sections (These are examples, replace with actual content as needed)
\newpage
\section{Introduction}
\subsection{Purpose}
Students\&Companies (S\&C) is a dynamic platform designed to connect university students seeking internships with companies offering valuable opportunities. By leveraging students' skills, experiences, and career interests alongside the specific needs and offerings of companies, S\&C aims to create seamless matches. The platform provides a recommendation system that notifies students about relevant internships and informs companies of suitable candidates. It also facilitates the selection process, supports feedback exchange, and helps both parties refine their profiles for better alignment. S\&C results useful thanks to the offered tools for monitoring internship progress and resolving issues collaboratively.
\subsubsection{Goals}
\begin{itemize}
  \item \textbf{G1} The companies can post on the platform the internships they want to advertise.
  \item \textbf{Requirement} Students can share their CVs on the platform
  \item \textbf{G2} The students can proactively look for internships posted by the companies on the platform.
  \item \textbf{G3} The students are suggested internships by the platform based on their CVs.
  \item \textbf{G5} The companies get information on some students that could interest them based on their CVs.
  \item \textbf{G1} Matches registered companies and students based on the information provided by internship advertisements and students' resumes.
  \item \textbf{G2} Allows registered students to proactively querying S\&C to identify interesting companies.
   \item \textbf{G3} Allows to start the assessment process for internship candidates after mutual approval, manages interview scheduling for companies, and finalizes candidate selections.
   \item \textbf{G4} Allows all sides to monitor the internship process and collects data from feedbacks and suggestions.
    \item \textbf{G5} Provides a recommendation to show an interesting profile to both sides.
    \item \textbf{G6} Provides suggestions to registered students and companies for enhancing internship advertisements and resumes
    \item \textbf{G7} Enables registered universities to monitor the internship process, handle complaints, and interrupt the internship when necessary.

\end{itemize}
\subsection{Scope}
\subsubsection{World Phenomena}
    \begin{itemize}
        \item \textbf{W1} A company wants to advertise an internship. 
        \item \textbf{W2} A student wants to find an internship opportunity.
        \item \textbf{W3} A student wants to discover interesting companies.
        \item \textbf{W4} A company wants to interview the internship candidate and select them.
        \item \textbf{W5} A student wants to evaluate an internship process
        \item \textbf{W6} A company and a student want to improve the internship advertisement and the CV, respectively.
        \item \textbf{W7} A University wants to follow an internship process
    \end{itemize}
\subsubsection{Shared Phenomena}
    \textit{Controlled by World}
    \begin{itemize}
        \item \textbf{S1} A student searches for internships on the platform.
        \item \textbf{S2} A company posts an internship advertisement on the platform.
        \item \textbf{S3} A student selects and accepts internships he wants to make contact with.
        \item \textbf{S4} A company selects and accepts a number between all the interested and recommended students as candidates for the given internship.
        \item \textbf{S5} A Student or a Company sends information of any type about the state of an on-going internship, like complaining or providing general information.
        \item \textbf{S6} A University interrupts an internship of a student after some complaints from the company or from the student. 
        \item \textbf{S7} A user registers either as Student, Company or University.
        
        \item \textbf{S1} A registered student submit their resumes on the S\&C.
        \item \textbf{S2} A registered company advertises an internship opportunity on S\&C.
        \item \textbf{S3} A company selects a student through the S\&C based on interesting skills from the student's CV. 
        \item \textbf{S4} A student selects the internship they want to apply for through the S\&C.
        \item \textbf{S5} A company creates interview request through the S\&C for the interview, after the match.
        \item \textbf{S6} A student and a company are able to provide feedback on their experiences, after the selection process.
        \item \textbf{S7} A company offers an internship position to the relevant student through the S\&C.
        \item \textbf{S8} A student accepts or declines an internship position through the S\&C.
        \item \textbf{S9} A student's university follows up on feedback about the internship process through the S\&C and is able to interrupt the internship if necessary.
        \item \textbf{S10} A student updates their resume based on the suggestions received from the S\&C.
        \item \textbf{S11} A student updates their advertisement based on the suggestions received from the S\&C.
    \end{itemize}
    \textit{Controlled by Machine}
    \begin{itemize}
        \item \textbf{S12} The system notifies a company about an interesting student resume.
        \item \textbf{S13} The system notifies to a student an internship that might interest him is available.
        \item \textbf{S14} The system starts a contact process when both an internship and a student confirm an interest in each other.
        \item \textbf{S15} The system shows some information about a student profiles to a company.
        \item \textbf{S16} The system shows some information about an internship advertisement and a company profile to a student.
        \item \textbf{S17} The system notifies a student when the interview date is scheduled.
        \item \textbf{S18} The system creates an interview link and sends to the both sides.
        \item \textbf{S19} The system analyzes the feedback data and provides suggestions for improving the profiles. 
        \item \textbf{S20} The system recommends interesting profiles and shows to the both sides.
        \item \textbf{S21} The system notifies a student when the company's decision is announced.
        \item \textbf{S22} The system creates a link for following and handling the internship process and sends to the student's university.
        \item \textbf{S23} The system notifies a student when internship is advertised from a favorite company which has manually searched for.
    \end{itemize}
\subsection{Definitions, Acronyms, Abbreviations}

\subsubsection{Definitions}
\subsubsection{Acronyms}
\subsubsection{Abbreviations}
\subsection{Revision history}
\subsection{Reference Documents}
\subsection{Document Structure}

\section{Overall Description}
\subsection{Product Perspective}
\subsubsection{Scenarios}
\begin{itemize}
    \item 1st Scenario: Signing up. User John has accessed the opening page; he registers into the site as a student, filling the required data and is sent back to the opening page.
    \item 2nd Scenario: Logging in. User John is in the opening page; he logs into the site using his credentials, and can now access to his possible operations.
    \item 3rd Scenario: Finalizing registration. The student John opens his profile page and uploads his CV document into the platform; he studies at Polime, so he connects his university mail with his account.
    \item 4th Scenario: Starting an offer. The company Emazon posts an internship offer on the platform, deciding the expiration date and the duration of it. Then Emazon lists the tasks the student will have to perform, the application domain and other relevant things regarding the internship.  
    \item 5th Scenario: Platform recommendation. Upon CV uploading, the platform analyzes John's CV and automatically suggests him all the current potential interesting internship offers. After Emazon has posted its internship offer, it gets notified to him too.
    \item 6th Scenario: Searching Internships. John opens the page of the available internships. He filters out the ones without benefits, and is left with few options. He applies for Guggl's offer.
    \item 7th Scenario: Student accepts recommendation. John gets the
    notification from Emazon and decides to accept it.
    \item 8th Scenario: Company accepts students. After Emazon posted the internship offer, it gets recommended some students by the platform, while some other students found the internship by manual searching. Emazon accepts John and some other students.
    \item 9th Scenario: Entering the selection. After the expiration date passes, the company starts the selection process. Emazon proposes the interviews dates and time schedules through the dedicated interface to all the contacted students.
    \item 10th Scenario: Student chooses interview day. Since John has established contact with Emazon, he selects the time and day of the interview through the dedicated interface.
    \item 11th Scenario: Student joins the interview. When the scheduled day for the interview comes, John can meet the interviewer using the link posted by the company on the platform.
    \item 10th Scenario: Writing Observations. Emazon writes a complaint in the provided space about John's current behaviour.
    \item 11th Scenario: University acts. The University Polime gets a complaint from Emazon about John's behaviour. Polime decides to interrupt his internship.
    \item 12th Scenario: Giving Feedback. At the end of the internship, John and Emazon fill a questionnaire about their experience with the platform and give suggestions to it.
\end{itemize}
\subsubsection{Class Diagram}
\subsubsection{State Charts}
\subsection{Product Functions}
\subsubsection{Requirements}
\subsection{User characteristics}
\subsection{Assumptions, dependencies and constraints}
\subsubsection{Domain Assumptions}
\subsubsection{Dependencies}


\section{Specific Requirements}
\subsection{External Interface Requirements}
\subsubsection{User Interfaces}
\subsubsection{Hardware Interfaces}
\subsubsection{Software Interfaces}
\subsubsection{Communication Interfaces}
\subsection{Functional Requirements}
\subsubsection{Use Case Diagrams}
\subsubsection{Use cases}
\subsubsection{mapping}
\subsection{Performance Requirements}
\subsection{Design Constraints}
\subsubsection{Standards compliance}
\subsubsection{Hardware limitations}
\subsection{Software System Attributes}
\subsubsection{Reliability and Availability}
\subsubsection{Security}
\subsubsection{Maintainability}
\subsubsection{Portability}

\section{Formal Analysis Using Alloy}
\subsection{Objectives of the Analysis}
\subsection{Alloy Code}

\section{Effort Spent}
\subsection{Effort Spent per Unit}

\section{References}
\subsection{References and Tools}

\maketitle



\end{document}






