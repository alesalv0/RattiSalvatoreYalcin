\documentclass{article}
\usepackage{graphicx} % Required for inserting images

\begin{document}

% Title Page
\title{RASD Document}
\author{Alessandro Salvatore, Erdal Yalçin, Leonardo Ratti}
\date{Academic Year: 2024-25}
\maketitle

% Table of Contents
\newpage
\tableofcontents

% Sections (These are examples, replace with actual content as needed)
\newpage
\section{Introduction}
\subsection{Purpose}
Students\&Companies (S\&C) is a dynamic platform designed to connect university students seeking internships with companies offering valuable opportunities. By leveraging students' skills, experiences, and career interests alongside the specific needs and offerings of companies, S\&C aims to create seamless matches. The platform provides a recommendation system that notifies students about relevant internships and informs companies of suitable candidates. It also facilitates the selection process, supports feedback exchange, and helps both parties refine their profiles for better alignment. S\&C results useful thanks to the offered tools for monitoring internship progress and resolving issues collaboratively.
\subsubsection{Goals}
\begin{itemize}
  \item \textbf{G1} The companies can post on the platform the internships they want to advertise.
  \item \textbf{G2} The students can proactively look for internships posted by the companies on the platform.
  \item \textbf{G3} The students are suggested internships by the platform based on their CVs.
  \item \textbf{G5} The companies get information on some students that could interest them based on their CVs.
  \item \textbf{G1} Matches registered companies and students based on the information provided by internship advertisements and students' resumes.
  \item \textbf{G2} Allows registered students to proactively querying S\&C to identify interesting companies.
   \item \textbf{G3} Allows to start the assessment process for internship candidates after mutual approval, manages interview scheduling for companies, and finalizes candidate selections.
   \item \textbf{G4} Allows all sides to monitor the internship process and collects data from feedbacks and suggestions.
    \item \textbf{G5} Provides a recommendation to show an interesting profile to both sides.
    \item \textbf{G6} Provides suggestions to registered students and companies for enhancing internship advertisements and resumes using collected feedback data.
    \item \textbf{G7} Enables registered universities to monitor the internship process, handle complaints, and interrupt the internship when necessary.

\end{itemize}
\subsection{Scope}
\subsubsection{World Phenomena}
    \begin{itemize}
        \item \textbf{W1} A company wants to advertise an internship. 
        \item \textbf{W2} A student wants to find an internship opportunity.
        \item \textbf{W3} A student wants to discover interesting companies.
        \item \textbf{W4} A company wants to interview the internship candidate and select them.
        \item \textbf{W5} A student wants to evaluate an internship process
        \item \textbf{W6} A company and a student want to improve the internship advertisement and the CV, respectively.
        \item \textbf{W7} A University wants to follow an internship process
    \end{itemize}
\subsubsection{Shared Phenomena}
    \textit{Controlled by World}
    \begin{itemize}
        \item \textbf{S1} A student searches for internships on the platform
        \item \textbf{S2} A company posts an internship offer on the platform.
        \item \textbf{S3} A student selects and accepts internships he wants to make contact with.
        \item \textbf{S4} A company selects and accepts a number between all the interested and recommended students as candidates for the given internship
        \item \textbf{S5} A Student or a Company sends information of any type about the state of an on-going internship, like complaining or providing general information.
        \item \textbf{S6} A University interrupts an internship of a student after some complaints from the company or from the student. 
        \item \textbf{S7} A user registers either as Student, Company or University.
        \item \textbf{S8} A registered students submit their resumes on the platform.
        \item \textbf{S9} A student and a company are able to provide feedback on their experiences.
        \item \textbf{S10}
    \end{itemize}
    \textit{Controlled by Machine}
    \begin{itemize}
        \item \textbf{S11} The system notifies to a student an internship that might interest him is available.
        \item \textbf{S12} The system starts a contact process when both an internship and a student confirm an interest in each other.
        \item \textbf{S13} The system notifies a student when the interview date is scheduled.
        \item \textbf{S14} The system notifies a student when the company's decision is announced.
        \item \textbf{S15} The system analyzes the feedback data and provides suggestions. 
        \item \textbf{S16} The system shows interesting profiles to the both sides.
        \item \textbf{S17} The system shows some information about a student profiles to a company.
        \item \textbf{S18} The system shows some information about an internship advertisement and a company profile to a student.
        \item \textbf{S19}
        \item \textbf{S20}
    \end{itemize}
\subsection{Definitions, Acronyms, Abbreviations}
\subsubsection{Definitions}
\subsubsection{Acronyms}
\subsubsection{Abbreviations}
\subsection{Revision history}
\subsection{Reference Documents}
\subsection{Document Structure}

\section{Overall Description}
\subsection{Product Perspective}
\subsubsection{Scenarios}
\subsubsection{Class Diagram}
\subsubsection{State Charts}
\subsection{Product Functions}
\subsubsection{Requirements}
    \item \textbf{Requirement} Students can share their CVs on the platform
\subsection{User characteristics}
\subsection{Assumptions, dependencies and constraints}
\subsubsection{Domain Assumptions}
\subsubsection{Dependencies}


\section{Specific Requirements}
\subsection{External Interface Requirements}
\subsubsection{User Interfaces}
\subsubsection{Hardware Interfaces}
\subsubsection{Software Interfaces}
\subsubsection{Communication Interfaces}
\subsection{Functional Requirements}
\subsubsection{Use Case Diagrams}
\subsubsection{Use cases}
\subsubsection{mapping}
\subsection{Performance Requirements}
\subsection{Design Constraints}
\subsubsection{Standards compliance}
\subsubsection{Hardware limitations}
\subsection{Software System Attributes}
\subsubsection{Reliability and Availability}
\subsubsection{Security}
\subsubsection{Maintainability}
\subsubsection{Portability}

\section{Formal Analysis Using Alloy}
\subsection{Objectives of the Analysis}
\subsection{Alloy Code}

\section{Effort Spent}
\subsection{Effort Spent per Unit}

\section{References}
\subsection{References and Tools}

\maketitle



\end{document}






